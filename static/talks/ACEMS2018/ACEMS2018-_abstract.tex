% !TeX program = pdfLaTeX
\documentclass[12pt]{article}
\usepackage{amsmath}
\usepackage{graphicx,psfrag,epsf}
\usepackage{enumerate}
\usepackage{natbib}
\usepackage{textcomp}
\usepackage[hyphens]{url} % not crucial - just used below for the URL
\usepackage{hyperref}
\providecommand{\tightlist}{%
  \setlength{\itemsep}{0pt}\setlength{\parskip}{0pt}}

%\pdfminorversion=4
% NOTE: To produce blinded version, replace "0" with "1" below.
\newcommand{\blind}{0}

% DON'T change margins - should be 1 inch all around.
\addtolength{\oddsidemargin}{-.5in}%
\addtolength{\evensidemargin}{-.5in}%
\addtolength{\textwidth}{1in}%
\addtolength{\textheight}{1.3in}%
\addtolength{\topmargin}{-.8in}%

%% load any required packages here




\begin{document}


\def\spacingset#1{\renewcommand{\baselinestretch}%
{#1}\small\normalsize} \spacingset{1}


%%%%%%%%%%%%%%%%%%%%%%%%%%%%%%%%%%%%%%%%%%%%%%%%%%%%%%%%%%%%%%%%%%%%%%%%%%%%%%

\if0\blind
{
  \title{\bf Outlier Detection in Non-Stationary Data Streams}

  \author{
        Priyanga Dilini Talagala \\
    Department of Econometrics and Business Statistics, Monash University,
    Australia\\
     and \\     Rob J Hyndman \\
    Department of Econometrics and Business Statistics, Monash University,
    Australia\\
     and \\     Kate Smith-Miles \\
    School of Mathematics and Statistics, University of Melbourne, Australia\\
      }
  \maketitle
} \fi

\if1\blind
{
  \bigskip
  \bigskip
  \bigskip
  \begin{center}
    {\LARGE\bf Outlier Detection in Non-Stationary Data Streams}
  \end{center}
  \medskip
} \fi

\bigskip
\begin{abstract}
This work develops a framework for detecting outlying series within a
large collection of time series in the context of non-stationary
streaming data. We define an outlier as an observation that is very
unlikely given the recent distribution of a given system. In this work
we make two fundamental contributions. First, we propose a framework
that provides early detection of anomalous behaviour within a large
collection of streaming time series data using extreme value theory.
Second, we propose a novel approach for early detection of
non-stationarity (also called ``concept drift'' in the machine learning
literature.) The proposed algorithm uses time series features as inputs,
and a density-based comparison to detect any significant change in the
distribution of the features. Using various synthetic and real world
datasets, we demonstrate the wide applicability and usefulness of our
proposed framework. This framework is implemented in the open source R
package \emph{oddstream}. We show that the proposed algorithm can work
well in the presence of noisy non-stationarity data within multiple
classes of time series.
\end{abstract}

\noindent%
{\it Keywords:} Anomaly Detection; Extreme value theory; Time series features;
Kernel-based density estimation; Nonstationary temporal data.
\vfill

\newpage
\spacingset{1.45} % DON'T change the spacing!



\bibliographystyle{agsm}
\bibliography{}

\end{document}
